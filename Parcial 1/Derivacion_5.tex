\documentclass{article}

\usepackage{geometry}

\usepackage{amsmath}

\usepackage{graphicx}

\title{Ejercicio 5 (Teórico)}

\author{Juan M. Perea, Andres Lemus}

\date{06/09/2023}

\begin{document}

\maketitle


Dados los polinomios:\\
		$f(x+h)=f(x)+hf'(x)+\frac{h^2}{2!}f''(x)+....\frac{h^n}{n!}\frac{d^n}{dx}f(x)$\\
		$f(x-h)=f(x)-hf'(x)+\frac{h^2}{2!}f''(x)-....(-1)^n\frac{h^n}{n!}\frac{d^n}{dx}f(x)$\\
		$f(x+2h)=f(x)+2hf'(x)+\frac{(2h)^2}{2!}f''(x)+....\frac{(2h)^n}{n!}\frac{d^n}{dx}f(x)$\\
		$f(x-2h)=f(x)-2hf'(x)+\frac{(2h)^2}{2!}f''(x)-....(-1)^n\frac{(2h)^n}{n!}\frac{d^n}{dx}f(x)$\\
		Se usa la siguiente combinación:\\
		$f(x+2h)+f(x-2h)-4[f(x+h)+f(x+h)]$\\
		$=-6f(x)+\frac{h^4}{2}f^{iv}(x).....$\\
		Se despeja la cuarta derivada:\\
		$f^{iv}(x)=\frac{f(x+2h)-4f(x+h)-6f(x)-4f(x-h)+f(x+2h)}{h^4}+[Oh^6]$\\
		Para algun punto de la partición:\\
		$f^{iv}(x_i)=\frac{f(x_{i+2})-4f(x_{i+1})-6f(x_i)-4f(x_{i-1})+f(x_{i-2})}{h^4}$
\end{document}