\documentclass{article}

\usepackage{geometry}

\usepackage{amsmath}

\usepackage{graphicx}

\title{Taller 1 (Teórico)}

\author{Juan M. Perea, Andres Lemus}

\date{29/08/2023}

\begin{document}

\maketitle

\begin{enumerate}
\renewcommand{\theenumi}{\roman{enumi}}
\item Derivación
	\begin{enumerate}
	\renewcommand{\theenumii}{\arabic{enumii}}
	\item
		Dados los polinomios de taylor:\\
		$f(x+2h)=f(x)+2hf'(x)+\frac{(2h)^2}{2!}f''(x)+....\frac{(2h)^n}{n!}\frac{d^n}{dx}f(x)$\\
		$f(x-2h)=f(x)-2hf'(x)+\frac{(2h)^2}{2!}f''(x)-....(-1)^n\frac{(2h)^n}{n!}\frac{d^n}{dx}f(x)$\\
		Se suman las expresiones tal que:\\
		$f(x+2h)+f(x-2h)=2f(x)+(2h)^2f''(x)+\frac{2(2h)^4}{4!}\frac{d^4}{dx}f(x)+\frac{2(2h)^6}{6!}\frac{d^6}{dx}f(x)....$\\
		Se despeja la segunda derivada:\\
		$f''(x)=\frac{f(x+2h)-2f(x)+f(x-2h)}{(2h)^2}+\frac{2(2h)^4}{4!}\frac{d^4}{dx}f(x)+\frac{2(2h)^6}{6!}\frac{d^6}{dx}f(x)....$\\
		Para un solo termino:\\
		$f''(x_i)\approx\frac{f(x_{i+2})-2f(x_i)+f(x_{i-2})}{4h^2}$
	\item
	\end{enumerate}
\item Interpolacion
\end{enumerate}

\end{document}